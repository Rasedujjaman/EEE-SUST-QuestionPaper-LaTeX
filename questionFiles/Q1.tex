%%%%%%%%%%%%%%%%%%%%%%%%%%%%%%%%%%%%%%%%%%%%%%%%%%%%%%%%%%%%%%%%%%%%%%%%%%%%%%%%%%%%%%%%%%%%%%%%%%%%%
%%%%%%%%%%%%%%%PART-A(Question#01- Frequency Response of Amplifier //%%%%%%%%%%%%%%%%%%%%%%%%%%
%%%%%%%%%%%%%%%%%%%%%%%%%%%%%%%%%%%%%%%%%%%%%%%%%%%%%%%%%%%%%%%%%%%%%%%%%%%%%%%%%%%%%%%%%%%%%%%%%%%%%	
\question 
\begin{parts}
%%%%%%%%%%%%%%%%%%%%%%%%%%%%%%%%%%%%%%%%%%%%%%%%%%%%%%%%%%%%%%%%%%%%%%%%%%%%%%%%%% 
\part[2] \color{red}If you want to put two or more figures side by side then you can use the latex minipage environment. \color{black} What is super Node? What is the difference between Super Mesh \& Mesh?

\begin{solution}

\end{solution}	
%%%%%%%%%%%%%%%%%%%%%%%%%%%%%%%%%%%%%%%%%%%%%%%%%%%%%%%%%%%%%%%%%%%%%%%%%%%%%%%%%%	
\part[4]Using nodal analysis, determine the potential across the \SI{4}{\ohm} resistor in Fig.~\ref{fig:ckt-nodal-analysis} 

\begin{figure}[H]
\centering
%%%%%%%%%%%%%%%%%%%%%%%%%%%%%%%%%%%%%%%%%%%%%%%%%%%%%%%%%%%%%%%%%%%
% First figure
\begin{minipage}[t]{0.48\textwidth}
\centering
\scalebox{0.7}{
\begin{tikzpicture}
% Paths, nodes and wires:
	\draw (4, 4) to[american resistor, l_={$2~\Omega$}, label distance=0.02cm] (7, 4);
	\draw (7, 4) to[american resistor, l_={$2~\Omega$}, label distance=0.02cm] (10, 4);
	\draw (10, 4) to[american resistor, l={$4~\Omega$}, label distance=0.02cm] (10, 1);
	\draw (4, 4) to[american resistor, l={$2~\Omega$}, label distance=0.02cm] (4, 1);
	
	
	
	\draw (4, 4) to[american resistor, l={$5~\Omega$}, label distance=0.02cm] (7, 6);
	\draw (7, 6) to[american resistor, l={$5~\Omega$}, label distance=0.02cm] (10, 4);
	\draw (7, 1) to[american current source, l_={$3~A$}, label distance=0.02cm] (7, 4);
	\draw (4, 1) -- (10, 1);
	\draw node[ground] at (7, 1) {};
	\draw node[ground] at (7, 1) {};
	\draw node[ground] at (7, 1) {};
	\draw node[ground] at (7, 1) {};
\end{tikzpicture}
}
\caption{}
\label{fig:ckt-nodal-analysis}
\end{minipage}
%%%%%%%%%%%%%%%%%%%%%%%%%%%%%%%%%%%%%%%%%%%%%%%%%%%%%%%%%%%%%%%%%%%
\hfill
%%%%%%%%%%%%%%%%%%%%%%%%%%%%%%%%%%%%%%%%%%%%%%%%%%%%%%%%%%%%%%%%%%%
% Second figure
\begin{minipage}[t]{0.48\textwidth}
\centering
\scalebox{0.7}{
\begin{tikzpicture}

% Paths, nodes, and wires:
	\draw (5, 7) to[american resistor, l={$2~\Omega$}, label distance=0.02cm] (5, 4);
	\draw (5, 4) to[american resistor, l={$4~\Omega$}, label distance=0.02cm] (5, 1);
	\draw (8, 7) to[american resistor, l={$8~\Omega$}, label distance=0.02cm] (8, 4);
	\draw (8, 4) to[american controlled voltage source, l={$2~v_o$}, label distance=0.02cm] (8, 1);
	\draw (5, 4) to[american current source, l={$3~A$}, label distance=0.02cm] (8, 4);
	\draw (3, 7) to[american voltage source, l_={$12~V$}, label distance=0.02cm] (3, 1);
	\draw (3, 1) -- (8, 1);
	\draw (3, 7) -- (8, 7);


    \node at (4.5, 5.5) [align=center] {\textbf{+} \\[0.25cm] $v_o$ \\[0.25cm] \textbf{-}}; % Increased spacing above and below V_2


	% Loop current labels:
	\node at (4.0, 4.1) {$i_1$}; % Left loop
	\node at (6.7, 5.6) {$i_2$}; % Middle loop
	\node at (6.7, 2.6) {$i_3$}; % Right loop

	% Loop current direction arrows:
	\draw[->] (3.8, 4.5) arc[start angle=120, end angle=-150, radius=0.5]; % i1 loop arrow
	\draw[->] (6.5, 6) arc[start angle=120, end angle=-150, radius=0.5]; % i2 loop arrow
	\draw[->] (6.5, 3) arc[start angle=120, end angle=-150, radius=0.5]; % i3 loop arrow
\end{tikzpicture}
}



\caption{}
\label{fig:ckt-mesh-analysis}
\end{minipage}
%%%%%%%%%%%%%%%%%%%%%%%%%%%%%%%%%%%%%%%%%%%%%%%%%%%%%%%%%%%%%%%%%%%

\end{figure}

\begin{solution}

\end{solution}
%%%%%%%%%%%%%%%%%%%%%%%%%%%%%%%%%%%%%%%%%%%%%%%%%%%%%%%%%%%%%%%%%%%%%%%%%%%%%%%%%%%%%	
\bdpart[2+2] Use mess analysis to find currents and voltage $v_o$ in the circuit of Fig.~\ref{fig:ckt-mesh-analysis}
	
\begin{solution}

\end{solution}	
%%%%%%%%%%%%%%%%%%%%%%%%%%%%%%%%%%%%%%%%%%%%%%%%%%%%%%%%%%%%%%%%%%%%%%%%%%%%%%%%%%%%%
\end{parts}
