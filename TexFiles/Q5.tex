%%%%%%%%%%%%%%%%%%%%%%%%%%%%%%%%%%%%%%%%%%%%%%%%%%%%%%%%%%%%%%%%%%%%%%%%%%%%%%%%%%%
%%%%%%%%%%%%%%%PART-B(Question#05-->
%%%%%%%%%%%%%%%%%%%%%%%%%%%%%%%%%%%%%%%%%%%%%%%%%%%%%%%%%%%%%%%
\question 
\begin{parts} 	
%%%%%%%%%%%%%%%%%%%%%%%%%%%%%%%%%%%%%%%%%%%%%%%%%%%%%%%%%%%%%%%%%%%%%%		
\part[$5$]For the circuit in Fig.~\ref{fig:ckt-inductor-acanalysis}, 
	 $i(t) = 4(2-e^{-10t})~\unit{\milli\ampere}$. If $i_2(t) = -1~\unit{\milli\ampere}$, find\droppoints
	
		\begin{subparts}
			\subpart $i_1(t)$
			\subpart $v(t),~v_1(t)$ and $v_2(t)$
			\subpart $i_1(t)$ and $i_2(t)$.
		\end{subparts}
		
		
\begin{figure}[H]
\centering

\begin{tikzpicture}
	% Paths, nodes, and wires:
	\draw (8.99, 3) to[cute inductor, l={$12~H$}, label distance=0.02cm] (8.99, 1);
	\draw (7, 3) to[cute inductor, l_={$4~H$}, label distance=0.02cm] (7, 1);
	\draw (5, 3) to[cute inductor, l={$2~H$}, label distance=0.02cm] (7, 3);
	\draw (5, 3) -- (4, 3);
	\draw (8.99, 1) -| (4, 1);
	\draw (7, 3) -| (8.99, 3);
	\draw node[ocirc] at (4, 3) {};
	\draw node[ocirc] at (4, 1) {};

	% Shorter current arrows with labels
	\draw[->, very thin] (4.2, 3.15) --++ (0.4, 0) node[midway, above] {\scriptsize $i$};  % Current for 2H inductor
	\draw[->, very thin] (7.3, 2.9) --++ (0, -0.4) node[anchor=south west] {\scriptsize $i_1$}; % Current for 4H inductor (shifted downward)
	\draw[->, very thin] (9.3, 2.9) --++ (0, -0.4) node[anchor=south west] {\scriptsize $i_2$}; % Current for 12H inductor (shifted downward)

	% Voltage label for the 12mH inductor with increased spacing
	\node at (8, 1.9) [align=center] {\textbf{+} \\[0.2cm] $v_2$ \\[0.2cm] \textbf{-}};
    % Voltage label for at the input terminal
	\node at (4, 1.9) [align=center] {\textbf{+} \\[0.2cm] $v$ \\[0.2cm] \textbf{-}};

    \node at (5.9, 2.6) { \textbf{+~~$v_1$~~-} };

\end{tikzpicture}
\caption{}
\label{fig:ckt-inductor-acanalysis}
\end{figure}


\begin{solution}

\end{solution}
%%%%%%%%%%%%%%%%%%%%%%%%%%%%%%%%%%%%%%%%%%%%%%%%%%%%%%%%%%%%%%%%%%%%%%	
\part[$3$]Find $i_x$ in the circuit of Fig.~\ref{fig:ckt-nodal-ac-analysis} using nodal analysis.\droppoints
		
		
\begin{figure}[H]
\centering
\begin{tikzpicture}
	% Paths, nodes, and wires:
	\draw (3, 3) to[american resistor, l={$10~\Omega$}, label distance=0.02cm] (5, 3);
	\draw (5, 3) to[cute inductor, l={$1~H$}, label distance=0.02cm] (7, 3);
	\draw (8.99, 3) to[cute inductor, l={$0.5~H$}, label distance=0.02cm] (8.99, 1);
	\draw (7, 1) to[american controlled current source, l_={$2i_x$}, label distance=0.02cm, mirror] (7, 3);
	\draw (5, 3) to[capacitor, l={$0.1~F$}, label distance=0.02cm] (5, 1);
	\draw (3, 3) to[american voltage source, l_={$20\cos 4t~V$}, label distance=0.02cm] (3, 1);
	\draw node[ground] at (7, 1) {};
	\draw (7, 3) -| (8.99, 3);
	\draw (3, 1) -| (8.99, 1);

	% Shifted current arrow and label (left and down)
	\draw[<-, very thin] (4.85, 2.3) --++ (0, 0.4) node[midway, left] {\scriptsize $i_x$};

\end{tikzpicture}



\caption{}
\label{fig:ckt-nodal-ac-analysis}
\end{figure}

\begin{solution}

\end{solution}
%%%%%%%%%%%%%%%%%%%%%%%%%%%%%%%%%%%%%%%%%%%%%%%%%%%%%%%%%%%%%%%%%%%%%%	
\part[$3$]Determine the currents $I_1$, $I_2$ and $I_{D2}$ for the network of Fig.~\ref{fig:ckt-diode-dcanalysis}.\droppoints
		
\begin{figure}[H]
\centering

\begin{tikzpicture}
	% Paths, nodes, and wires:
	\draw (5, 6) to[empty diode, l_={$D1$}, label distance=0.02cm] 
	    node[above left=0.3cm] {\small Si} (8, 6); % Shifted "Si" label to the left of D1
	    
	\draw (8, 6) to[empty diode, l={$D2$}, i>={$i_{D2}$}, label distance=0.02cm] 
	    node[above left=0.3cm] {\small Si} (8, 4); % Shifted "Si" label to the left of D2
	
	\draw (8, 6) to[american resistor, l={$R_1$}, i>={$i_1$}, label distance=0.02cm] (11, 6);
	\draw (8, 4) to[american resistor, l_={$R_2$}, i>={$i_2$}, label distance=0.02cm] (5, 4);
	\draw (5, 6) to[battery1, l_={$E=20~V$}, label distance=0.02cm] (5, 4);
	\draw (8, 4) -| (11, 6);
\end{tikzpicture}
\caption{}
\label{fig:ckt-diode-dcanalysis}
\end{figure}


\begin{solution}

\end{solution}
%%%%%%%%%%%%%%%%%%%%%%%%%%%%%%%%%%%%%%%%%%%%%%%%%%%%%%%%%%%%%%%%%%%%%%				
\end{parts}
